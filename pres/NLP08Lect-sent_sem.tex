% !TEX TS-program = pdflatex
% !TeX program = pdflatex
% !TEX encoding = UTF-8
% !TEX spellcheck = en_US

\documentclass[xcolor=table]{beamer}

\usepackage{../extra/beamer/karimnlp}

\input{options}

\subtitle[08- Sentence semantics]{Chapter 08\\Sentence semantics} 

\changegraphpath{../img/sent-sem/}

\begin{document}
	
\begin{frame}
	\frametitle{\inserttitle}
	\framesubtitle{\insertshortsubtitle: Introduction}

	\begin{exampleblock}{Example of sentences}
		\begin{center}
			\Large\bfseries
			My cat caught a mouse with its claws.
			
			My cat caught a mouse with pleasure.
			
			My cat caught a mouse with another cat.
		\end{center}
	\end{exampleblock}
	
	\begin{itemize}
		\item What instrument did the cat use to catch the mouse? How? With who?
		\item The last sentence means: 
		\begin{itemize}
			\item My cat caught two animals: a mouse and another mouse.
			\item The two cats caught a mouse.
		\end{itemize}
	\end{itemize}

\end{frame}

\begin{frame}
	\frametitle{\inserttitle}
	\framesubtitle{\insertshortsubtitle: Some humor}
	
	\begin{center}
		\vgraphpage{humor/humor-ambiguity.jpeg}
	\end{center}
	
\end{frame}

\begin{frame}
	\frametitle{\inserttitle}
	\framesubtitle{\insertshortsubtitle: Plan}

	\begin{multicols}{2}
	%	\small
	\tableofcontents
	\end{multicols}

\end{frame}

%===================================================================================
\section{Semantic roles}
%===================================================================================

\begin{frame}
	\frametitle{\insertshortsubtitle}
	\framesubtitle{\insertsection}
	
	\begin{center}
		Blank page 
		
		...
		
		Blank like my mind trying to think what to put in here
	\end{center}
	
\end{frame}

\subsection{Thematic roles}

\begin{frame}
	\frametitle{\insertshortsubtitle: \insertsection}
	\framesubtitle{\insertsubsection}
	
	\begin{table}
		\tiny\bfseries
		\begin{tblr}{
				colspec = {p{.13\textwidth}p{.42\textwidth}p{.35\textwidth}},
				row{odd} = {lightblue},
				row{even} = {lightyellow},
				row{1} = {darkblue},
				rowsep=2pt,
			}
			\textcolor{white}{Role} & \textcolor{white}{Description} & \textcolor{white}{Example}\\
			
			AGENT &
			The volitional causer of an event &
			\expword{\underline{Karim} broke the window with a rock.}\\
			
			EXPERIENCER & 
			The experiencer of an event & 
			\expword{\underline{Karim} has a headache.}\\
			
			FORCE &
			The non-volitional causer of the event &
			\expword{\underline{The wind} blows the debris.}\\
			
			THEME &
			The participant most directly affected by an event &
			\expword{Karim broke \underline{the window} with a rock.}\\
			
			RESULT &
			The end product of an event &
			\expword{The city built \underline{a baseball field}.}\\
			
			CONTENT &
			The proposition or content of a propositional event &
			\expword{Sara asked\newline \underline{``Did you meet Leila in a supermarket?"}}\\
			
			INSTRUMENT &
			An instrument used in an event &
			\expword{\underline{a rock} broke the window.}\\
			
			BENEFICIARY &
			The beneficiary of an event &
			\expword{Sara makes hotel reservations for \underline{her boss}.}\\
			
			SOURCE &
			The origin of the object of a transfer event &
			\expword{I came from \underline{Jijel}.}\\
			
			GOAL &
			The destination of an object of a transfer event &
			\expword{I went to \underline{Algiers}.}\\
			
			LOCATIVE & 
			The specification of the place where the action or event designated by the predicate takes place &
			\expword{I live in \underline{Jijel}.}\\
		\end{tblr}
		\caption{Some thematic roles, adapted from \cite{2019-jurafsky-martin}}
	\end{table}
	
\end{frame}

\subsection{FrameNet}

\begin{frame}
	\frametitle{\insertshortsubtitle: \insertsection}
	\framesubtitle{\insertsubsection}

	\begin{minipage}{.68\textwidth}
		\begin{itemize}
			\item {\scriptsize \url{https://framenet.icsi.berkeley.edu/fndrupal/}}
			\item {\scriptsize \url{https://www.nltk.org/howto/framenet.html}}
			\item Resource used to represent meaning
			\item based on``\keyword{Frame semantics}" theory of \keyword{Fillmore}
			\item annotated manually
			\item Many semantic roles
		\end{itemize}
	\end{minipage}
	\begin{minipage}{.3\textwidth}
		\hgraphpage{frameNet-logo.jpg}
	\end{minipage}
	
	\begin{exampleblock}{Example of sentences having the same frame}
		The price of petrol increased.
		
		The price of petrol rose.
		
		There has been a rise in the price of petrol.
	\end{exampleblock}

\end{frame}

\begin{frame}
	\frametitle{\insertshortsubtitle: \insertsection}
	\framesubtitle{\insertsubsection: Structure}
	
	\begin{itemize}
		\item \optword{Frame}: schematic relation of a situation.
		\begin{itemize}
			\item name, definition, semantic type, frame elements, lexical units, examples and relations with other frames
		\end{itemize}
	
		\item \optword{Frame Element (FE)}: a semantic role specific to a frame. 
		\begin{itemize}
			\item It describes a participant or a situation of the frame.
			\item Two types of FE: important (\keyword{Core}) and secondary (\keyword{Non-Core}).
		\end{itemize}
	
		\item \optword{Lexical Units (LU)}: lemmas with their PoS.
		\begin{itemize}
			\item A lexical unit triggers the frame when encountered.
		\end{itemize}
	
		\item \optword{Relations}: inter-frames relations.
		\begin{itemize}
			\item Example, \expword{inheritance}.
		\end{itemize}
	\end{itemize}
	
\end{frame}

\begin{frame}
	\frametitle{\insertshortsubtitle: \insertsection}
	\framesubtitle{\insertsubsection: Example of lexical units}

	\vspace{-12pt}
	\begin{table}
		\scriptsize\bfseries
		\begin{tblr}{
				colspec = {p{.15\textwidth}p{.3\textwidth}p{.45\textwidth}},
				row{odd} = {lightblue},
				row{even} = {lightyellow},
				row{1} = {darkblue},
			}
		
			\textcolor{white}{Lexical Unit} & \textcolor{white}{Frame} & \textcolor{white}{Example}\\
	
			break.n & Opportunity & \\	
			break.v & Cause\_harm & \expword{Jolosa broke a rival player's jaw.}\\
			break.v & Compliance & \expword{He broke his promess.}\\
			break.v & Experience\_bodily\_harm & \expword{I broke my arm in the accident.}\\
			break.v & Cause\_to\_fragment & \expword{Michael broke the bottle against his head}\\
			break.v & Render\_nonfunctional & \expword{I guess I broke the doorknob by twisting it too hard.}\\
			break.v & Breaking\_off & \expword{The handle broke off of the pot.}\\
			break.v & Breaking\_apart & \expword{The handle broke off of the pot.}\\
	
		\end{tblr}
		\caption{Sematic frames activated by `break" LU.}
	\end{table}

\end{frame}

\begin{frame}
	\frametitle{\insertshortsubtitle: \insertsection}
	\framesubtitle{\insertsubsection: Example of semantic frame (1)}
	
	
	\begin{table}
		\tiny\bfseries
		\begin{tblr}{
				colspec = {p{.15\textwidth}p{.75\textwidth}},
				row{odd} = {lightblue},
				row{even} = {lightyellow},
				row{1,3,8} = {darkblue},
				rowsep=0pt,
				cell{1,3,8}{1} = {c=2}{c},
			}
		
			\textcolor{white}{Cause\_to\_fragment} & \\
			
			Definition & An \textcolor{red}{Agent} suddenly and often violently separates the \textcolor{red}{Whole\_patient} into two or more smaller \textcolor{red}{Pieces}, resulting in the \textcolor{red}{Whole\_patient} no longer existing as such. Several lexical items are marked with the semantic type Negative, which indicates that the fragmentation is necessarily judged as injurious to the original \textcolor{red}{Whole\_patient}. Compare this frame with Damaging, Render\_non-functional, and Removing. \\	
			
%			\rowcolor{darkblue}
			\textcolor{white}{FEs (Core)} & \\
			
			Agent [Agt] \newline \textcolor{blue}{Semantic Type: Sentient} & 
			The conscious entity, generally a person, that performs the intentional action that results in the \textcolor{red}{Whole\_patient} being broken into \textcolor{red}{Pieces}. \newline \expword{\underline{I and I alone} can SHATTER the gem and break the curse.} \\
			
			Cause [cau] & 
			An event which leads to the fragmentation of the \textcolor{red}{Whole\_patient}. \\
			
			Pieces [Pieces]	& 
			The fragments of the \textcolor{red}{Whole\_patient} that result from the \textcolor{red}{Agent}'s action.
			\newline
			\expword{I SMASHED the toy boat to \underline{flinders}.} \\
			
			Whole\_patient [Pat] & The entity which is destroyed by the \textcolor{red}{Agent} and that ends up broken into \textcolor{red}{Pieces}.
			\newline
			\expword{Shattering someone's confidence is a little different than SHATTERING \underline{a dish}.} \\
			
%			\rowcolor{darkblue}
			\textcolor{white}{FEs (None-Core)} & \\
			
			Degree [Degr] \newline \textcolor{blue}{Semantic Type: Degree} &
			The degree to which the fracturing is completed. 
			\newline
			\expword{I SHATTERED the vase \underline{completely}.} \\
			
			Explanation [Exp] \newline \textcolor{blue}{Semantic Type: State\_of\_affairs} &
			A state of affairs that the Agent is responding to in performing the action. \newline
			\expword{He TORE the treaty UP out of frustration.}
			
		\end{tblr}
		\caption{Example of frame ``Cause\_to\_fragment" (part 1)
			\newline
			{\tiny\url{ https://framenet2.icsi.berkeley.edu/fnReports/data/frameIndex.xml?frame=Cause_to_fragment}}%
		}
	\end{table}
	
\end{frame}

\begin{frame}
	\frametitle{\insertshortsubtitle: \insertsection}
	\framesubtitle{\insertsubsection: Example of semantic frame (2)}
	
	\begin{table}
		\tiny\bfseries
		\begin{tblr}{
				colspec = {p{.15\textwidth}p{.75\textwidth}},
				row{odd} = {lightblue},
				row{even} = {lightyellow},
				row{1,3,7,11} = {darkblue},
				rowsep = 0pt,
				cell{1,3,6,7,11}{1} = {c=2}{c},
			}
			\textcolor{white}{Cause\_to\_fragment} & \\
			
			 & \\	
			 
			
%			\rowcolor{darkblue}
			\textcolor{white}{FEs (None-Core)} & \\
			
			Explanation [Exp] \newline \textcolor{blue}{Semantic Type: State\_of\_affairs} &	
			A state of affairs that the \textcolor{red}{Agent} is responding to in performing the action.
			\newline
			\expword{He TORE the treaty UP \underline{out of frustration}.} \\
			
			Instrument [Ins] \newline \textcolor{blue}{Semantic Type: Physical\_entity} &
			An entity directed by the  \textcolor{red}{Agent} that interacts with a \textcolor{red}{Whole\_patient} to accomplish its fracture. \\
			
			
			{\large ...} & \\
			
%			\rowcolor{darkblue}
			\textcolor{white}{Frame-frame Relations} & \\
			
			Inherits from & Transitive\_action \\
			Uses & Destroying \\
			Is Causative of & Breaking\_apart \\
			
%			\rowcolor{darkblue}
			\textcolor{white}{Lexical Units} & \\
			
			& break apart.v, break down.v, break up.v, break.v, chip.v, cleave.v, dissect.v, dissolve.v, fracture.v, fragment.v, rend.v, rip up.v, rip.v, rive.v, shatter.v, shiver.v, shred.v, sliver.v, smash.v, snap.v, splinter.v, split.v, take apart.v, tear up.v, tear.v \\
			
		\end{tblr}
		\caption{Example of frame ``Cause\_to\_fragment" (part 2)
			\newline
			{\tiny\url{ https://framenet2.icsi.berkeley.edu/fnReports/data/frameIndex.xml?frame=Cause_to_fragment}}%
		}
	\end{table}
	
\end{frame}


\begin{frame}
	\frametitle{\insertshortsubtitle: \insertsection}
	\framesubtitle{\insertsubsection: Example of valence patterns}
	
	\vspace{-6pt}
	\begin{table}
		\tiny\bfseries
		\begin{tabular}{|p{.12\textwidth}|p{.12\textwidth}|p{.12\textwidth}|p{.12\textwidth}|p{.12\textwidth}|p{.12\textwidth}|}
			\hline
			\rowcolor{darkblue}
			\textcolor{white}{Number Annotated} & \multicolumn{5}{|l|}{\textcolor{white}{Patterns}}\\
			\hline
			\multicolumn{6}{l}{ }\\
			
			\hline
			\rowcolor{lightyellow}
			1 TOTAL & \textcolor{red}{Agent} & \textcolor{red}{Instrument} & \textcolor{red}{Pieces} & \textcolor{red}{Whole\_patient} & \\
			\hline
			\rowcolor{lightyellow}
			(1) & CNI \newline - - & PP[with] \newline Dep & INI \newline - - & NP \newline Ext & \\
			\hline
			\multicolumn{6}{l}{ }\\
			
			\hline
			\rowcolor{lightblue}
			1 TOTAL & \textcolor{red}{Agent} & \textcolor{red}{Means} & \textcolor{red}{Pieces} & \textcolor{red}{Time} & \textcolor{red}{Whole\_patient} \\
			\hline
			\rowcolor{lightblue}
			(1) & NP \newline Ext & 2nd \newline - - & INI \newline - - & Sinterrog \newline Dep & NP \newline Obj \\
			\hline
			\multicolumn{6}{l}{ }\\
			
			\hline
			\rowcolor{lightyellow}
			4 TOTAL & \textcolor{red}{Agent} & \textcolor{red}{Pieces} & \textcolor{red}{Whole\_patient} & & \\
			\hline
			\rowcolor{lightyellow}
			(4) & NP \newline Ext & INI \newline - - & NP \newline Obj & & \\
			\hline
		\end{tabular}
		\caption{Lexical entry of trigger ``fracture" (verb) of frame ``Cause\_to\_fragment": valence pattern list excerpt.}
	\end{table}
	
\end{frame}

\begin{frame}
	\frametitle{\insertshortsubtitle: \insertsection}
	\framesubtitle{\insertsubsection: Example of lexicographic annotations}
	
	\vskip-12pt
	\begin{figure}
		\begin{tcolorbox}[colback=white, colframe=blue, boxrule=1pt, text width=.91\textwidth]
			\tiny\bfseries
			\vspace{-6pt}
		\begin{itemize}
			\item 429-s20-rcoll-skull
			\begin{enumerate}\tiny
				\item \ [\textsubscript{\color{red}Agent} Former England Under-21 player Keith Benton] FRACTURED\textsuperscript{\color{red}Target} [\textsubscript{\color{red}Whole\_patient} his son Seb 's skull] [\textsubscript{\color{red}Time} when he hit the ball into the crowd during a match in Buckingham]. [\textsubscript{\color{red}Pieces} INI] 
				\item \ [\textsubscript{\color{red}Agent} He] hit a lamp-post and FRACTURED\textsuperscript{\color{red}Target} [\textsubscript{\color{red}Whole\_patient} Mike 's skull]. [\textsubscript{\color{red}Pieces} INI] 
				\item When he found the man [\textsubscript{\color{red}Agent} he] threw the acid into his face and beat him with the hammer , FRACTURING\textsuperscript{\color{red}Target} [\textsubscript{\color{red}Whole\_patient} his skull] and his thumb. [\textsubscript{\color{red}Pieces} INI] 
				\item \ [\textsubscript{\color{red}Agent} A nanny] has been jailed after FRACTURING\textsuperscript{\color{red}Target} [\textsubscript{\color{red}Whole\_patient} the skulls of two new born babies in her care]. [\textsubscript{\color{red}Pieces} INI] 
			\end{enumerate}
			\item 520-s20-np-vping
			\item 620-s20-np-ppother
			\item 660-s20-trans-simple
			\begin{enumerate}\tiny
				\item \ [\textsubscript{\color{red}Agent} Then 17-year-old Lee Diaz, of North End Gardens, Bishop Auckland], attacked a second party-goer, Carl Gent, punching him in the face and FRACTURING\textsuperscript{\color{red}Target} [\textsubscript{\color{red}Whole\_patient} his jaw]. [\textsubscript{\color{red}Pieces} INI] 
			\end{enumerate}
			
%			\item 670-s20-pass-by
			\item 680-s20-pass
			\begin{enumerate}\tiny
				\item \ [\textsubscript{\color{red}Whole\_patient} It] was FRACTURED\textsuperscript{\color{red}Target} [\textsubscript{\color{red}Instrument} with a solvent-cleaned chisel], and the outer orange layer discarded. [\textsubscript{\color{red}Agent} CNI][\textsubscript{\color{red}Pieces} INI] 
			\end{enumerate}
			
%			\item 690-s20-trans-other
%			\item 730-s20-ppwith
%			\item 780-s20-ppother
%			\item 880-s20-intrans-simple
%			\item 890-s20-intrans-adverb
%			\item 900-s20-other
		\end{itemize}\vspace{-6pt}
	\end{tcolorbox}\vspace{-6pt}

		\caption{Lexicographic annotations excerpt of the trigger ``fracture" (verb) of the frame ``Cause\_to\_fragment".}
	\end{figure}
	
\end{frame}

\subsection{PropBank}

\begin{frame}
	\frametitle{\insertshortsubtitle: \insertsection}
	\framesubtitle{\insertsubsection}

	
	%\begin{minipage}{.68\textwidth}
	\begin{itemize}
		\item {\scriptsize \url{https://propbank.github.io/}}
		\item {\scriptsize \url{https://www.nltk.org/howto/propbank.html}}
		\item Annotated corpus based on Predicate-Arguments structure
		\item Less semantic roles 
		\begin{itemize}
			\item \optword{Proto-Agent}
			\begin{itemize}
				\item voluntarily participate in an event or state
				\item cause an event or a state change of another participant
			\end{itemize}
			\item \optword{Proto-Patient}
			\begin{itemize}
				\item experience a change of state
				\item be affected by another participant
			\end{itemize}
		\end{itemize}
	\end{itemize}
	
\end{frame}

\begin{frame}
	\frametitle{\insertshortsubtitle: \insertsection}
	\framesubtitle{\insertsubsection: Structure}
	
	\begin{itemize}
		\item \optword{Roleset id}: Verbs are annotated by their meanings.
		\begin{itemize}
			\item E.g. \expword{know.01: be cognizant of, realize ; know.02: be familiar with, have experienced}		
		\end{itemize}
		
		\item \optword{Roles}: Each verb/meaning has a set of possible arguments.  
		\begin{itemize}
			\item \optword{Arg0}: PROTO-AGENT
			\item \optword{Arg1}: PROTO-PATIENT
			\item \optword{Arg2}: in general; beneficiary, instrument, attribute, or end state.
			\item \optword{Arg3}: in general; start point, beneficiary, instrument, or attribute.
			\item \optword{Arg4}: end point
			\item Arg2 ... Arg5 are not consistent in the corpus
		\end{itemize}
	
		\item \optword{Modifiers}: Marked by \keyword{ArgM}  
		\begin{itemize}
			\item \optword{ArgM-TMP}: When?
			\item \optword{ArgM-LOC}: Where?
			\item \optword{ArgM-MNR}: How? 
			\item ...
		\end{itemize}
		
		\item \optword{Annotated examples} 
	\end{itemize}
	
\end{frame}

\begin{frame}
	\frametitle{\insertshortsubtitle: \insertsection}
	\framesubtitle{\insertsubsection: Example of predicates}
	

	\begin{figure}
	%	\tiny
		\scriptsize
		\begin{tcolorbox}[colback=white, colframe=blue, boxrule=1pt, text width=.91\textwidth]
		\begin{itemize}
			\item \textbf{Roleset id}
			\begin{itemize}\scriptsize
				\item \textbf{know.01}: be cognizant of, realize
			\end{itemize}
			\item \textbf{Roles}
			\begin{itemize}\scriptsize
				\item \textbf{Arg0}: knower
				\item \textbf{Arg1}: fact that is known
				\item \textbf{Arg2}: entity that arg1 is known ABOUT
			\end{itemize}
		
			\item \textbf{Example: know-v: sentential thing known}
			\begin{itemize}\scriptsize
				\item \ [\textsubscript{\color{red}Arg0} The other side] knows [\textsubscript{\color{red}Arg1} that Giuliani has always been prochoice].
			\end{itemize}
		
			\item \textbf{Example: know-v: attributive}
			\begin{itemize}\scriptsize
				\item \ [\textsubscript{\color{red}Arg0} He] did[\textsubscript{\color{red}ArgM-NEG} n't] know [\textsubscript{\color{red}Arg1} (anything)] [\textsubscript{\color{red}Arg2} about most of the cases] [\textsubscript{\color{red}ArgM-TMP} until Wednesday].
			\end{itemize}
		\end{itemize}
		\end{tcolorbox}
			
		\caption{ProBank annotation excerpt of predicate ``know" \url{http://verbs.colorado.edu/propbank/framesets-english-aliases/know.html}}
	\end{figure}
	
\end{frame}


%===================================================================================
\section{Semantic roles labeling (SRL)}
%===================================================================================

\begin{frame}
	\frametitle{\insertshortsubtitle}
	\framesubtitle{\insertsection\ (Example)}

	\begin{figure}
		\hgraphpage{exp-srl_.pdf}
		\caption{Example of ProbBank-based SRL [\url{https://demo.allennlp.org/semantic-role-labeling/}]}
	\end{figure}
	
\end{frame}

\begin{frame}
	\frametitle{\insertshortsubtitle}
	\framesubtitle{\insertsection}
	
	\begin{itemize}
		\item Rule-based / Linguistic Approaches.
		
		e.g. \expword{"NP [subject] + verb + NP [object]" → Agent + Verb + Patient}
		
		\item Statistical / Feature-based Machine Learning
		
		\item Neural-based
		\begin{itemize}
			\item RNN-based
			\item Transformer-based
		\end{itemize}
	\end{itemize}
	
\end{frame}

\subsection{Features-based}

\begin{frame}
	\frametitle{\insertshortsubtitle: \insertsection}
	\framesubtitle{\insertsubsection}
	
	\begin{itemize}
		\item \optword{Idea}: browse the parse tree and classify the nodes.
		\begin{itemize}
			\item \optword{Input}: features on the node (phrase) and the predicate.
			\item \optword{Output}: classes are those of FrameNet or PropBank + \keyword{None} to mark a node with no role
		\end{itemize} 
		\item The classification process can be optimized
		\begin{enumerate}
			\item \optword{Pruning}: eliminate classification constituents using heuristics.
			\item \optword{Identification}: classify the nodes in \keyword{Argument} or \keyword{None}.
			\item \optword{Classification}: classes are those of FrameNet or PropBank.
		\end{enumerate} 
	\end{itemize}
	
\end{frame}

\begin{frame}
	\frametitle{\insertshortsubtitle: \insertsection}
	\framesubtitle{\insertsubsection: Example of annotated parse tree}
	
	\vspace{-0.2cm}
	\begin{figure}
		\hgraphpage{srl-tree.pdf}
		\caption{Example of a parse tree with a path from a phrase to the main predicate \cite{2019-jurafsky-martin}}
	\end{figure}
	
\end{frame}

\begin{frame}
	\frametitle{\insertshortsubtitle: \insertsection}
	\framesubtitle{\insertsubsection: Some features}

	\begin{itemize}
		\item Sentence's main predicate. E.g. \expword{Using computers \underline{enhances} work.};
		\item Phrase type. E.g. \expword{NP, S, PP};
		\item Phrase head (main) word. E.g. \expword{The extremely good \underline{weather}}; 
		\item Head word's PoS. E.g. \expword{The extremely good weather/\underline{NOUN}};
		\item Path from the current node to the predicate. E.g. \expword{NP\textuparrow S\textdownarrow VP \textdownarrow VPD};
		\item Voice: active or passive;
		\item Relative position to the predicate: before or after;
		\item ...
	\end{itemize}
	
\end{frame}

\subsection{RNN-based}

\begin{frame}
	\frametitle{\insertshortsubtitle: \insertsection}
	\framesubtitle{\insertsubsection}

	\begin{minipage}{.48\textwidth}
		\begin{itemize}
			\item \optword{Input}: Words embeddings + predicate indicator (predicate or not).
			\item \optword{Output}: PropBank senses probabilities.  
			\item Using \keyword{IOB} notation
		\end{itemize}
	\end{minipage}
	\begin{minipage}{.5\textwidth}
		\begin{figure}
			\hgraphpage{srl-lstm.pdf}
			\caption{SRL system using LSTM \cite{2017-he-al}}
		\end{figure}
	\end{minipage}
	
\end{frame}

\subsection{Transformer-based}

\begin{frame}
	\frametitle{\insertshortsubtitle: \insertsection}
	\framesubtitle{\insertsubsection\ \cite{2023-mohammadshahi-henderson}}
	\begin{figure}
		\centering
		\vgraphpage[0.7\textheight]{SynG2G-Tr_.pdf}
		\caption{Architecture of SynG2G-Tr \cite{2023-mohammadshahi-henderson}}
	\end{figure}
	
\end{frame}

\begin{frame}[fragile]
	\frametitle{\insertshortsubtitle: \insertsection}
	\framesubtitle{\insertsubsection\ \cite{2023-chanin}}
	
	3 subtasks as sequence-to-sequence tasks performed by T5 model
	\begin{itemize}
		\item Trigger identification
		{\scriptsize\bfseries\color{olivegreen} 
		\begin{verbatim}
				input: "TRIGGER: It was no use trying the lift."
				output: "It was no use *trying the *lift."
		\end{verbatim}
		}
		\item Frame classification
		{\scriptsize\bfseries\color{olivegreen} 
		\begin{verbatim}
				input: "FRAME Body_movement Building_subparts Cause_motion Cause_to_end 
				        Connecting_architecture Theft: It was no use trying the *lift."
				output: "Connecting_architecture"
		\end{verbatim}
		}
		\item Argument extraction
		{\scriptsize\bfseries\color{olivegreen} 
		\begin{verbatim}
				input: "ARGS Connecting_architecture | Part Connected_locations Creator 
				        Descriptor Direction Goal Material Orientation Source Whole: 
				        It was no use trying the *lift."
				output: "Part='the lift'"
		\end{verbatim}
		}
	\end{itemize}
	
	\begin{verbatim}
		>> pip install frame-semantic-transformer
	\end{verbatim}
	
\end{frame}

\begin{frame}
	\frametitle{\insertshortsubtitle: \insertsection}
	\framesubtitle{\insertsubsection\ \cite{2020-kalyanpur}}
	
	\begin{minipage}{.58\textwidth}
		\small
		\begin{tabular}{|p{4cm}|p{2cm}|}
			\hline
			Input & Output \\
			\hline
			FRAME: 0 Two 1 of 2 the 3 cast 4
			fainted 5 and 6 most 7 of 8 the 9 rest
			10 * repaired * 11 to 12 the 13 nearest
			14 bar 15 . & Self motion\\
			&\\
			ARGS for Self motion: 0 Two 1 of 2 the
			3 cast 4 fainted 5 and 6 most 7 of 8 the
			9 rest 10 * repaired * 11 to 12 the 13
			nearest 14 bar 15 . & Self mover = 6-9 $|$ Goal = 11-14 $|$\\
			\hline
		\end{tabular}
	\end{minipage}
	\begin{minipage}{.4\textwidth}
		\begin{figure}
			\hgraphpage{Kalyanpur2020_.pdf}
			\caption{Multi-task model for semantic parsing \cite{2020-kalyanpur}}
		\end{figure}
	\end{minipage}
	
\end{frame}

%===================================================================================
\section{Semantic representation of sentences}
%===================================================================================

\begin{frame}
	\frametitle{\insertshortsubtitle}
	\framesubtitle{\insertsection}
	
	\begin{itemize}
		\item There are several ways to express the same meaning.
		\begin{itemize}
			\item \expword{The student has prepared a report.}
			\item \expword{A report has been prepared by the student.}
		\end{itemize}
	
		\item Sentences in natural languages can be ambiguous
		\begin{itemize}
			\item \expword{John floated the boat between the rocks.}
			\item See more: \url{https://plato.stanford.edu/entries/ambiguity/}
		\end{itemize}
	
		\item Multiple tasks can benefit from semantic representation:
		\begin{itemize}
			\item Natural language understanding (NLU)
			\item Question-Answering
			\item Information retrieval (IR)
			\item Machine translation
			\item Automatic text summarization
		\end{itemize}
	\end{itemize}
	
\end{frame}

\subsection{First order logic}

\begin{frame}
	\frametitle{\insertshortsubtitle: \insertsection}
	\framesubtitle{\insertsubsection}
	
	\begin{itemize}
		\item \optword{term}: an object.
		\begin{itemize}
			\item \optword{constant}: a specific object
			%
			\expword{Karim, ESI, Algeria}
			
			\item \optword{function}: return an object based on others
			%
			\expword{LocationOf(ESI)}
			
			\item \optword{variable}: anonymous object 
			%
			\expword{x, y, z}
			
		\end{itemize}
		\item \optword{predicate}: a relation 
		
		\expword{University(ESI), TeacherAt(Karim, ESI)}
		
		\item \optword{connective}: AND ($ \wedge $), OR ($ \vee $), NOT ($ \neg $ ), IMPLY ($\rightarrow$) and EQUIVALENT ($ \Leftrightarrow $)
		
		\expword{TeacherAt(Karim, ESI) $\wedge$ University(ESI)}
		
		\item \optword{quantifier}: EXISTS ($\exists$) and ALL ($\forall$)
		
		\expword{I eat at a restaurant near ESI.}
		
		\expword{$\exists$ x Restaurant(x) $\wedge$ Near(LocationOf(x), LocationOf(ESI)) $\wedge$  Eat(Interlocutor, x)}
	\end{itemize}
	
\end{frame}

\begin{frame}
	\frametitle{\insertshortsubtitle: \insertsection}
	\framesubtitle{\insertsubsection: Formalism}
	
	\begin{figure}
%		\vgraphpage[0.6\textheight]{LPO-gram_.pdf}
		\scriptsize
		\begin{tabular}{rcl}
			\hline\hline
			Formula & \textrightarrow & AtomicFormula \\
			        & \textbar        & Formula Connective Formula \\
			        & \textbar        & Quantifier Variable, ... Formula \\
			        & \textbar        & $\textlnot$ Formula \\
			        & \textbar        & (Formula) \\
			AtomicFormula & \textrightarrow & Predicate (Term, ...) \\
			Term    & \textrightarrow & Function(Term, ...) \\
			        & \textbar        & Constant \\
			        & \textbar        & Variable \\
			Connective & \textrightarrow & $\wedge$ \textbar $\vee$ \textbar $\Rightarrow$ \\
			Quantifier & \textrightarrow & $\forall$ \textbar $\exists$ \\
			Constant & \textrightarrow & ESI \textbar Karim \textbar Algérie ...\\
			Variable & \textrightarrow & x \textbar y \textbar ... \\
			Predicate & \textrightarrow & Ecole \textbar EnseignatA \textbar Utiliser \textbar ... \\
			Function & \textrightarrow & EmplacementDe \textbar ... \\
			\hline\hline
		\end{tabular}
		\caption{A grammar specifying first order logic syntax \cite{2019-jurafsky-martin} (Adapted from \cite{2002-russell-norvig})}
	\end{figure}
	
\end{frame}

\begin{frame}
	\frametitle{\insertshortsubtitle: \insertsection}
	\framesubtitle{\insertsubsection: Some humor}
	
	\begin{center}
		\vgraphpage{humor/humor-FOL.jpeg}
	\end{center}
	
\end{frame}

\subsection{Graphs (AMR)}

\begin{frame}[fragile]
	\frametitle{\insertshortsubtitle: \insertsection}
	\framesubtitle{\insertsubsection}
	
	\begin{itemize}
		\item Abstract Meaning Representation \cite{2013-banarescu-al}
		\item A rooted, labeled, directed and acyclic graph
		\item It uses \keyword{PropBank} concepts
	\end{itemize}

	\begin{exampleblock}{Threes formats: Example ``The boy wants to go"}
		\begin{minipage}{.3\textwidth}
			\optword{Logic format}
			
			\footnotesize
			$ \exists $ w, b, g : 
			
			instance(w, want-01) 
			
			$ \wedge $ instance(g, go-01) 
			
			$ \wedge $ instance(b, boy) 
			
			$ \wedge $ arg0(w, b) 
			
			$ \wedge $ arg1(w, g) 
			
			$ \wedge $ arg0(g, b)
		\end{minipage}
		\begin{minipage}{.35\textwidth}
			\optword{AMR format}
			
			\begin{verbatim}
(w / want-01
    :arg0 (b / boy)
    :arg1 (g / go-01
    :arg0 b))
			\end{verbatim}
			
		\end{minipage}
		\begin{minipage}{.3\textwidth}
			\optword{Graph format}
			
			\hgraphpage{amr-graph-exp.pdf}
		\end{minipage}
	\end{exampleblock}
	
\end{frame}

\begin{frame}[fragile]
	\frametitle{\insertshortsubtitle: \insertsection}
	\framesubtitle{\insertsubsection: Relations}
	
	\begin{itemize}
		\item \optword{frame arguments}: following PropBank
		\begin{itemize}
			\item :arg0, :arg1, :arg2, :arg3, :arg4, :arg5
		\end{itemize}
		\item \optword{General semantic relations}
		\begin{itemize}
			\item :accompanier, :age, :beneficiary, :cause, :compared-to, :concession, :condition, :consist-of, :degree, :destination, :direction, :domain, :duration, :employed-by, :example, :extent, :frequency, :instrument, :li, :location, :manner, :medium, :mod, :mode, :name, :part, :path, :polarity, :poss, :purpose, :source, :subevent, :subset, :time, :topic, :value
		\end{itemize}
		\item \optword{Quantity relations}
		\begin{itemize}
			\item :quant, :unit, :scale
		\end{itemize}
		\item \optword{Date relations}
		\begin{itemize}
			\item :day, :month,
			:year, :weekday, :time, :timezone, :quarter,
			:dayperiod, :season, :year2, :decade, :century,
			:calendar, :era
		\end{itemize}
		\item \optword{List relations}
		\begin{itemize}
			\item :op1, :op2, :op3, :op4, :op5,
			:op6, :op7, :op8, :op9, :op10
		\end{itemize}
	\end{itemize}
	
\end{frame}

\begin{frame}[fragile]
	\frametitle{\insertshortsubtitle: \insertsection}
	\framesubtitle{\insertsubsection: Parsing \cite{2023-lee-amr}}
	
	\begin{minipage}{.40\textwidth}
	AMR graph:
	\bfseries\scriptsize
	\begin{verbatim}
		(r / reveal-01
		   :ARG0 (s / statistic)
		   :ARG1 (t / tend-02
		   :ARG1 (t2 / thing
		   :ARG1-of (i / invest-01
		   :ARG0 (c / country
		      :wiki "Taiwan"
		      :name (n / name
		      :op1 "Taiwan"))
		   :ARG2 (m / mainland)
		      :mod (b / business)))
		   :ARG2 (i2 / increase-01
		   :ARG1 t2))
		      :mod (a / also))
	\end{verbatim}
	\end{minipage}
	\begin{minipage}{.58\textwidth}
	Serialized AMR graph:
	\bfseries\scriptsize
	\begin{verbatim}
		( reveal-01 :ARG0 ( statistic ) :ARG1
		( tend-02 :ARG1 ( thing :ARG1-of ( invest-01
		:ARG0 ( country :name ( name :op1 "Taiwan" ) )
		:ARG2 ( mainland ) :mod ( business ) ) )
		:ARG2 ( increase-01 :ARG1 thing ) ) :mod
		( also ) )
	\end{verbatim}
	
	\normalfont\normalsize
	Input text with the task prefix (T5):
	\bfseries\scriptsize
	\begin{verbatim}
		amr generation ; Statistics also revealed
		that Taiwanese business investments in the
		mainland is tending to increase
	\end{verbatim}
	\end{minipage}
	
\end{frame}


\subsection{Semantic parsing}

\begin{frame}
	\frametitle{\insertshortsubtitle: \insertsection}
	\framesubtitle{\insertsubsection: Lambda notation}
	
	\begin{itemize}
		\item \keyword{$ \lambda $-Expression}
		\begin{itemize}
			\item Anonymous function
			\item \keyword{$ \lambda $x.P(x)}
			\item Example: \expword{$ \lambda $x.USE(x, BERT)} to express the clause \expword{x uses BERT}
		\end{itemize}
		
		\item \keyword{$ \lambda $-Reduction}
		\begin{itemize}
			\item Substitute a variable by an expression
			\item \keyword{$ \phi (\psi) $}
			\item Example: \expword{$ \lambda $x.USE(x, BERT)\ KARIM = USE(KARIM, BERT)}
		\end{itemize}
		
		\item For fun: \url{http://davidmarino.nfshost.com/lambdalinguist/index.html}
	\end{itemize}
	
\end{frame}

\begin{frame}
	\frametitle{\insertshortsubtitle: \insertsection}
	\framesubtitle{\insertsubsection: Logical representation generation}
	
	\begin{center}
		\small
		\begin{tabular}{llll}
			\hline\hline
			S  & \textrightarrow\ NP VP && VP.sem(NP.sem) \\
			VP & \textrightarrow\ V\textsubscript{t} NP && V\textsubscript{t}.sem(NP.sem)\\
			VP & \textrightarrow\ V\textsubscript{i} && V\textsubscript{i}.sem \\
			V\textsubscript{t}  & \textrightarrow\ uses && $ \lambda $y.$ \lambda $x.USE(x, y) \\
			V\textsubscript{i}  & \textrightarrow\ sleeps && $ \lambda $x.SLEEP(x) \\
			NP  & \textrightarrow\  Karim && KARIM \\
			NP  & \textrightarrow\  BERT && BERT \\
			\hline\hline
		\end{tabular}
		
		\hgraphpage[0.7\textwidth]{sem-tree.pdf}
	\end{center}
\end{frame}

\begin{frame}
	\frametitle{\insertshortsubtitle: \insertsection}
	\framesubtitle{\insertsubsection: Quantifiers}
	
	\begin{center}
		\small
		\begin{tabular}{llll}
			\hline\hline
			S  & \textrightarrow\ NP VP && NP.sem(VP.sem) \\
			DET & \textrightarrow\ each && $\lambda P.\lambda Q.\forall x (P(x) \Rightarrow Q(x))$ \\
			
			VP & \textrightarrow\ V\textsubscript{t} NP && V\textsubscript{t}.sem(NP.sem) \\
			V\textsubscript{t}  & \textrightarrow\ uses && $\lambda P.\lambda x.P(\lambda y.USE(x, y))$ \\
			
			VP & \textrightarrow\ V\textsubscript{i} && V\textsubscript{i}.sem \\
			V\textsubscript{i}  & \textrightarrow\ sleeps && $ \lambda $x.SLEEP(x) \\
			
			NP & \textrightarrow\ DET NN && DET.sem(NN.sem) \\
			NN  & \textrightarrow\  student && STUDENT \\
			
			NP & \textrightarrow\ NNP && $\lambda P.P(NNP.sem)$ \\
			NNP  & \textrightarrow\  Karim && KARIM \\
			
			DET & \textrightarrow\ a && $\lambda P.\lambda Q.\exists x\ P(x) \wedge Q(x)$ \\
			NNP  & \textrightarrow\  BERT && BERT \\
			\hline\hline
		\end{tabular}
	\end{center}
	
\end{frame}

\begin{frame}
	\frametitle{\insertshortsubtitle: \insertsection}
	\framesubtitle{\insertsubsection: Quantifiers (Example)}
	
	\begin{center}
		\hgraphpage[0.8\textwidth]{sem-qtree.pdf}
		\small
		\begin{align*}
			VP.sem & = V_t.sem(NP.sem) \\
			& = \textcolor{red}{\lambda P}.\lambda x.\textcolor{red}{P}(\lambda y.USE(x, y))(\textcolor{blue}{\lambda P.P(BERT)}) \\
			& = \lambda x.\textcolor{red}{\lambda P}.\textcolor{red}{P}(BERT)(\textcolor{blue}{\lambda y.USE(x, y)}) \\
			& = \lambda x.\textcolor{red}{\lambda y}.USE(x, \textcolor{red}{y})(\textcolor{blue}{BERT}) \\
			& = \lambda x.USE(x, BERT) \\
		\end{align*}
	\end{center}
	
\end{frame}

\begin{frame}[fragile]
	\frametitle{\insertshortsubtitle: \insertsection}
	\framesubtitle{\insertsubsection: NLTK example}
	
	\begin{exampleblock}{Example pf semantic parsing using NLTK}
		\optword{Code:}
		{\scriptsize
			\begin{lstlisting}[language=Python]
from nltk import load_parser
parser = load_parser('grammars/book_grammars/simple-sem.fcfg', trace=0)
sentence = 'Angus gives a bone to every dog'
tokens = sentence.split()
for tree in parser.parse(tokens):
print(tree.label()['SEM'])
			\end{lstlisting}
		}
		
		\optword{Result:}
		{\scriptsize\bfseries
			\begin{lstlisting}
all z2.(dog(z2) -> exists z1.(bone(z1) & give(angus,z1,z2)))
			\end{lstlisting}
		}
		
	\end{exampleblock}
	
	\begin{itemize}
		\item See: \url{https://www.nltk.org/book/ch10.html}
	\end{itemize}
	
\end{frame}

\begin{frame}
	\frametitle{\insertshortsubtitle: \insertsection}
	\framesubtitle{\insertsubsection: Some humor}
	
	\begin{center}
		\vgraphpage{humor/humor-sens.jpg}
	\end{center}
	
\end{frame}

\begin{frame}
	\frametitle{\insertshortsubtitle: \insertsection}
	\framesubtitle{\insertsubsection: Some humor (Reasoning)}
	
	\begin{center}\bfseries\large
		even chatGPT knows its limits
		
		\vgraphpage[0.7\textheight]{humor/chatgpt-solvr.png}
	\end{center}
	
\end{frame}


\insertbibliography{NLP08}{*}

\end{document}

