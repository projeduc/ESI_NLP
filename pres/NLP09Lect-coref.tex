% !TEX TS-program = pdflatex
% !TeX program = pdflatex
% !TEX encoding = UTF-8
% !TEX spellcheck = en_US

\documentclass[xcolor=table]{beamer}

\usepackage{../extra/beamer/karimnlp}
\input{options}

\subtitle[09- Coreference resolution]{Chapter 09\\Coreference resolution} 

\changegraphpath{../img/coref/}

\begin{document}
	
\begin{frame}
	\frametitle{\inserttitle}
	\framesubtitle{\insertshortsubtitle: Introduction}

	\begin{exampleblock}{Example of a paragraph}
		\begin{center}
			\Large\bfseries
		The cat saw a flower. \textbf{\color{red}It} smelled \textbf{\color{red}it}. \textbf{\color{red}It} has a very nice scent.
		\end{center}
	\end{exampleblock}
	
	\begin{itemize}
		\item Who smelled whom: the cat or the flower?
		\item Who has a nice scent: the cat or the flower?
	\end{itemize}

\end{frame}


\begin{frame}
	\frametitle{\inserttitle}
	\framesubtitle{\insertshortsubtitle: Plan}

	\begin{multicols}{2}
	%	\small
	\tableofcontents
	\end{multicols}

\end{frame}

%===================================================================================
\section{References}
%===================================================================================

\begin{frame}
	\frametitle{\insertshortsubtitle}
	\framesubtitle{\insertsection}
	
	\begin{block}{Reference [Cambridge dictionary]}
		the act of mentioning someone or something in speech or writing
	\end{block}
	
	\begin{block}{Reference [Merriam-Webster]}
		one referred to or consulted: such as
		\begin{itemize}
			\item  a person to whom inquiries as to character or ability can be made
			\item a statement of the qualifications of a person seeking employment or appointment given by someone familiar with the person
			\item a source of information (such as a book or passage) to which a reader or consulter is referred
			\item a work (such as a dictionary or encyclopedia) containing useful facts or information
		\end{itemize}
	\end{block}
	
\end{frame}

\subsection{References forms}

\begin{frame}
	\frametitle{\insertshortsubtitle: \insertsection}
	\framesubtitle{\insertsubsection}
	
	\begin{itemize}
		\item \optword{Pronouns}
		\begin{itemize}
			\item Personal: \expword{\underline{Karim} entered. \underline{He} started class.}
			\item Possessive: \expword{\underline{Karim} has started \underline{his} course.}
			\item ...
		\end{itemize}
	
		\item \optword{Nominal phrases}
		\begin{itemize}
			\item \expword{I have a little \underline{cat}. \underline{This animal} is very mean.}
		\end{itemize}
	
		\item \optword{Proper nouns} 
		\begin{itemize}
			\item \expword{The \underline{école nationale supérieure d'informatique} is located in Algiers. Like all Algerian universities, you must have the baccalaureate to study at \underline{ESI}.}
		\end{itemize}
	
		\item \optword{Zero anaphora}
		\begin{itemize}
			\item \expword{\underline{Karim} presented and \underline{$ \phi $} explained the course.}
			\item \expword{\begin{CJK}{UTF8}{min}\underline{カリムさん}はESIに行きます。\underline{$ \phi $} あそこに教えています。\end{CJK}}
		\end{itemize}
	\end{itemize}
	
\end{frame}

\subsection{Referencing manner}

\begin{frame}
	\frametitle{\insertshortsubtitle: \insertsection}
	\framesubtitle{\insertsubsection}
	
	\begin{itemize}
		\item \optword{Anaphora}
		\begin{itemize}
			\item a reference to a previous word/phrase is called \keyword{antecedent}.
			\item E.g. \expword{\underline{The lesson} is very long. \underline{It} will take longer.}
		\end{itemize}
	
		\item \optword{Cataphora}
		\begin{itemize}
			\item a reference to a following word/phrase is called \keyword{postcedent}.
			\item E.g. \expword{\underline{It} is very long, \underline{this course}!}
		\end{itemize}
	
		\item \optword{Shared antecedents}
		\begin{itemize}
			\item a reference to several words and/or phrases.
			\item E.g. \expword{\underline{The lesson} will be followed by \underline{an exercise}. \underline{They} are important to understand the subject.}
		\end{itemize}
	
		\item \optword{Coreference noun phrases}
		\begin{itemize}
			\item two noun phrases where each is a reference to the other.
			\item E.g. \expword{\underline{Some of my colleagues} have been really supportive. \underline{This kind of people} earns my gratitude.}
		\end{itemize}
	\end{itemize}
	
\end{frame}

\subsection{Properties of Coreference Relations}

\begin{frame}
	\frametitle{\insertshortsubtitle: \insertsection}
	\framesubtitle{\insertsubsection\ (1)}
	
	
	\begin{itemize}
		\item \optword{Number}: Singular, Dual, Plural.
		\begin{itemize}
			\item \expword{\underline{Lions} are chasing a \underline{dear}. \underline{They} are so fast.}
		\end{itemize}
	
		\item \optword{Person}: First, Second, Third.
		\begin{itemize}
			\item \expword{\underline{My brother}\textsubscript{1} and \underline{I}\textsubscript{2} have repaired \underline{his bicycle}\textsubscript{1} after \underline{mine}\textsubscript{2}.}
		\end{itemize}
		
		\item \optword{Gender}: Masculine, Feminine, Neutral.
		\begin{itemize}
			\item \expword{When the girl met her \underline{father}, \underline{he} was very happy.}
		\end{itemize}
		
		\item \optword{Binding theory's constraints}: syntactic constraints on the mention-antecedent relation.
		\begin{itemize}
			\item \expword{Janet bought herself a bottle of fish sauce.} [herself = Janet]
			\item \expword{Janet bought her a bottle of fish sauce.} [her $\ne$ Janet]
		\end{itemize}
	
	\end{itemize}
	
\end{frame}

\begin{frame}
	\frametitle{\insertshortsubtitle: \insertsection}
	\framesubtitle{\insertsubsection\ (2)}
	
	\begin{itemize}
		
		\item \optword{Recency}: entities mentioned more recently are more likely to be an antecedent.
		\begin{itemize}
			\item \expword{The doctor found an old map. Jim found an even older \underline{map}. \underline{It} is about an island.}
		\end{itemize}
		
		\item \optword{Grammatical relations}: subjects are more likely to be an antecedents than objects.
		\begin{itemize}
			\item \expword{\underline{Karim} went to a restaurant with his friend. \underline{He} asked for a couscous dish.}
		\end{itemize}
		
		\item \optword{Verb semantics}: the mention follows the emphasis of the verb (the one that caused the event).
		\begin{itemize}
			\item \expword{\underline{John} telephoned Bill. \underline{He} lost the laptop.}
			\item \expword{John criticized \underline{Bill}. \underline{He} lost the laptop.}
		\end{itemize}
	\end{itemize}
	
\end{frame}

\begin{frame}
	\frametitle{\insertshortsubtitle: \insertsection}
	\framesubtitle{\insertsubsection: Some humor}
	
	\begin{center}
		\vgraphpage{humor/humor-ref.jpg}
	\end{center}
	
\end{frame}

%===================================================================================
\section{Coreference resolution}
%===================================================================================

\begin{frame}
	\frametitle{\insertshortsubtitle}
	\framesubtitle{\insertsection}
	
	\hgraphpage{coref-arch.pdf}
	
	\begin{itemize}
		\item \optword{Mention detection}
		\item \optword{Linking models}
		\begin{itemize}
			\item Mention-Pair: each mention with others.
			\item Mention-Rank: a mention must be linked to only one antecedent.
			\item Entity-based: detect coreference clusters.
		\end{itemize}
		
		\item \optword{Coreference resolution approaches}
		\begin{itemize}
			\item Rules-based
			\item Features-based (manually engineered)
			\item Embedding-based
		\end{itemize}
	\end{itemize}
	
\end{frame}


\subsection{Mention detection}

\begin{frame}
	\frametitle{\insertshortsubtitle: \insertsection}
	\framesubtitle{\insertsubsection}
	
	\begin{itemize}
		\item \optword{Entity extraction}
		\begin{itemize}
			\item Nominal phrases, pronouns, named entities.
		\end{itemize}
	
		\item \optword{Entities filtering}
		\begin{itemize}
			\item Using linguistic rules. E.g. In the expression ``\expword{It is thought that ...}", the pronoun ``\expword{It}" is not a reference. In this case, a list of cognitive verbs \expword{believe, think, etc.} can be used to filter this pronoun.
			\item Using machine learning: one detector for references and another for antecedents.
		\end{itemize}
	
	\end{itemize}
	
\end{frame}


\begin{frame}
	\frametitle{\insertshortsubtitle: \insertsection}
	\framesubtitle{\insertsubsection: Rule-based \cite{2013-lee-al}}
	
	\begin{itemize}
		\item Delete a mention if there is another larger one with the same syntactic head. 
		E.g., ``\expword{the software company}" vs. ``\expword{the software company situated in Algiers}". 
		
		\item Discard numerical entities such as percentages, money, cardinals and quantities.
		E.g., ``\expword{9\%}", ``\expword{100\$}", ``\expword{one hundred nine}".
		
		\item Eliminate mentions that are part of another (quantifier + ``of"). 
		E.g., ``\expword{a total of 177 projects}", ``\expword{none of them}", ``\expword{millions of people}".
		
		\item Remove pronouns that are not used as a reference (pleonastic ``it"). 
		E.g., ``\expword{It is thought that ...}", the pronoun ``it" is not a reference. 
		A list of cognitive verbs \expword{believe, think, etc.} can be used to filter this pronoun.
		
		\item Eliminate adjectival forms of nations or national acronyms.
		E.g., ``\expword{American}", ``\expword{U.S.}", ``\expword{U.K.}".
		
	\end{itemize}
	
\end{frame}


\begin{frame}
	\frametitle{\insertshortsubtitle: \insertsection}
	\framesubtitle{\insertsubsection: NN-based \cite{2020-yu-al}}
	
	\begin{figure}[ht]
		\centering
		\hgraphpage[\textwidth]{mention-detection-arch.pdf}
		\caption[Three architectures to detecting mentions]{Three architectures to detecting mentions; Inspired from \cite{2020-yu-al}}
		\label{fig:det-mention-yu}
	\end{figure}
	
\end{frame}



\subsection{Linking}

\begin{frame}
	\frametitle{\insertshortsubtitle: \insertsection}
	\framesubtitle{\insertsubsection: Mention-Pair model}
	
	\begin{itemize}
		\item Detect if there is a coreference between two mentions.
		\item Binary classification (coreference, not coreference).
		\item \textbf{Problem}: contradictions. E.g. \expword{Ms Kennedy $ \leftarrow $ Kennedy, Kennedy $ \leftarrow $ He}
	\end{itemize}
	\begin{figure}
		\centering
		\hgraphpage[.8\textwidth]{mention-pair-exp.pdf}
		\caption{Example of Mention-Pair model \cite{2019-jurafsky-martin}}
	\end{figure}
	
\end{frame}

\begin{frame}
	\frametitle{\insertshortsubtitle: \insertsection}
	\framesubtitle{\insertsubsection: Mention-Rank model}
	
	\begin{itemize}
		\item Among previous mentions, decide which one is the antecedent.
		\item $ \hat{a} = \arg\max\limits_{a_i \in \{\epsilon, a_1, \ldots, a_{n-1}\}} P(a_i|a_n) $
		\item $ \epsilon $: means, there is no antecedent.
	\end{itemize}
	\begin{figure}
		\centering
		\hgraphpage{mention-rank-exp.pdf}
		\caption{Example of Mention-Rank model \cite{2019-jurafsky-martin}}
	\end{figure}
	
\end{frame}

\begin{frame}
	\frametitle{\insertshortsubtitle: \insertsection}
	\framesubtitle{\insertsubsection: Entity-based model}
	
	\begin{itemize}
		\item Also called: cluster-ranking model
		\item \textbf{Input}: two mentions' clusters
		\item verify if the two clusters are compatible
		\item Estimate if a cluster is antecedent to another such as in Mention-Rank
		\item If the two clusters represent the same mention, merge them
		\item \textbf{Output}: A set of clusters
	\end{itemize}
	
\end{frame}

\subsection{Evaluation}

\begin{frame}
	\frametitle{\insertshortsubtitle: \insertsection}
	\framesubtitle{\insertsubsection}
	
	\begin{itemize}
		\item \keyword{MUC}: \optword{Message Understanding Conference}
		\begin{itemize}
			\item Link-based metric
			\item Number of shared binary links between the reference and the system
		\end{itemize}
		\item \keyword{B\textsuperscript{3}}
		\begin{itemize}
			\item Mention-based metric
			\item Global R/P is calculated based on R/P of individual mentions.
		\end{itemize}
		\item \keyword{CEAF}: \optword{Constrained entity-alignment F-Measure}
		\begin{itemize}
			\item mention-based OR entity-based (2 versions depending on used similarity)
			\item A mention of the system is aligned with a single reference
		\end{itemize}
		\item \keyword{BLANC} 
		\begin{itemize}
			\item Link-based metric
			\item Global R/P is calculated based on R/P of coreference links and that of non coreference
		\end{itemize}
		\item \keyword{LEA}: \optword{Link based entity aware}
		\begin{itemize}
			\item Entity size as an importance measure
			\item Resolved coreferences' links are evaluated.
		\end{itemize}
	\end{itemize}
	
\end{frame}

\begin{frame}
	\frametitle{\insertshortsubtitle: \insertsection}
	\framesubtitle{\insertsubsection: Some humor}
	
	\begin{center}
		\vgraphpage{humor/humor-res.jpg}
	\end{center}
	
\end{frame}

%===================================================================================
\section{Related tasks}
%===================================================================================

\begin{frame}
	\frametitle{\insertshortsubtitle}
	\framesubtitle{\insertsection}
	
	\begin{itemize}
		\item Used tasks
		\begin{itemize}
			\item Preprocessing task (Chapter 3)
			\item PoS tagging (Chapter 5)
			\item Named entities recognition (Chapter 5)
		\end{itemize}
		\item Similar tasks
		\begin{itemize}
			\item Entity linking
			\item Citation attribution: find who said/wrote a discourse
		\end{itemize}
	\end{itemize}
	
\end{frame}

\subsection{Entity linking}

\begin{frame}
	\frametitle{\insertshortsubtitle: \insertsection}
	\framesubtitle{\insertsubsection}
	
	\begin{itemize}
		\item Associate a mention in a text with its representation in a structured knowledge base.
		\item useful to have in-depth knowledge on some entities in the real world.
		\item \keyword{Wikification}: link a mention with a Wikipédia page
		\item Entity linking steps: 
		\begin{itemize}
			\item \optword{Mention detection}: detect all knowledge base entities linked to a mention using queries.
			\item \optword{Mention disambiguation}: find the most likely entity using machine learning.
		\end{itemize}
		\item Some APIs: 
		\begin{itemize}
			\item \url{https://labs.tib.eu/falcon/}
			\item \url{https://dandelion.eu}
		\end{itemize}
	\end{itemize}
	
\end{frame}

\subsection{Named entity recognition (NER)}

\begin{frame}
	\frametitle{\insertshortsubtitle: \insertsection}
	\framesubtitle{\insertsubsection}
	
	\begin{itemize}
		\item locate and classify named entities in a text
		\item \keyword{Named entity}: persons, places, organizations, quantities, etc.
		\item a data extraction subtask
		\item Recognition techniques
		\begin{itemize}
			\item \optword{Rules}: use rules and lists of names to search for entities and detect their types.
			\item \optword{Features-based ML}: word embeddings, prefixes and suffixes, list membership, neighbors features, etc.
			\item \optword{Sequences labeling}: As seen in chapter 5 using \keyword{IOB} annotation.
			
			\expword{\scriptsize
			$ \underbrace{Google}_{B-ORG} $ 
			$ \underbrace{LLC}_{I-ORG} $ 
			$ \underbrace{is}_{O} $ 
			$ \underbrace{founded}_{O} $ 
			$ \underbrace{in}_{O} $ 
			$ \underbrace{Silicon}_{B-LOC} $ 
			$ \underbrace{Valley}_{B-LOC} $ 
			$ \underbrace{by}_{O} $ 
			$ \underbrace{Larry}_{B-PER} $ 
			$ \underbrace{Page}_{I-PER} $ 
			$ \underbrace{and}_{O} $ 
			$ \underbrace{Sergey}_{B-PER} $ 
			$ \underbrace{Brin}_{I-PER} $
			}
		\end{itemize}
	\end{itemize}
	
\end{frame}


\insertbibliography{NLP09}{*}

\end{document}

