% !TEX TS-program = pdflatex
% !TeX program = pdflatex
% !TEX encoding = UTF-8
% !TeX spellcheck = en_US

\documentclass{../../../extra/aakpract/aakpract}

\settitle{Parsing sentences using CKY}
\setversion{Lab02}
\setyear{2024-2025}
\setauthor{Mr. Abdelkrime Aries}

\begin{document}

\maketitle

\begin{center}
	\begin{minipage}{0.8\textwidth}
		\small
		In order to understand syntactic analysis, we want to implement CKY algorithm from scratch as well as the naive evaluation algorithm.
	\end{minipage}
\end{center}

\section{Implementation}

The algorithm is implemented in its entirety except for the functions to be completed.
There are two classes:
\begin{itemize}
	\item CKY: Its constructor has as arguments the Chomsky normal form grammar (NCF) and the lexicon (a word and the list of grammatical categories).
	It parses a sentence passed as a list of words.
	\item Syntax: It uses CKY class to apply the syntactic analysis.
	It generates a graphical syntax tree (based on the Dot language).
	Also, it evaluates our parsing model.
\end{itemize}

In this lab, you must implement the "parse" method seen in the lecture, but modify it to accept unitary rules.
Also, you must implement the method for comparing two trees.
{\color{red} The tree is represented as a tuple which is comparable since it is hashable. In our case, you have to implement the traverse algorithm.}

\subsection{Parsing}

The method takes as input a list of words (sentence).
It must use the FNC grammar (with unitary rules) and the lexicon that are introduced in the constructor.
It must return the analysis matrix seen in the lecture:
\begin{itemize}
	\item a list of lists: to represent the rows and columns
	\item a cell's content is a list: to represent the different possible rules
	\item a rule is a tuple (A, k, iB, iC): A is the variable, k is used to calculate the position of B (with iB) and C (with iC) of the rule $A \rightarrow B C$.
	If $A \rightarrow B$, there is no right part (A, k, iB, iC=-1).
\end{itemize}

\subsection{pr\_eval}

This function takes a reference tree (which is a node) and a system-generated tree.
It compares the two structures and returns (precision $P$, recall $R$).
\begin{itemize}
	\item TP is is the number of arcs traversed from the root in both trees (ref and sys).
	\item $|ref|$ and $|sys|$ are the numbers of arcs in ref and sys trees respectively
\end{itemize}

\[P = \frac{TP}{|sys|},\ \ \ R = \frac{TP}{|ref|}\]

Note: if one of the two trees exists and the other does not, the function should have (0, 0) as its result.
If both do not exist, the function should return (1, 1).


\section{Questions}

Answer these questions in the same file as the code.

\begin{enumerate}
	\item "If PoS tagging is used with CKY, there will be no ambiguities". True or False? Justify.
	\item "If PoS tagging is used with CKY, there will be no out-of-vocabulary problem". True or False? Justify.
\end{enumerate}

\section{Logistics}

This lab must be done in a team of no more than \textbf{2} students.
It takes \textbf{1h} (the assignment must be submitted at the end of the session).

\subsection{Evaluation}

\begin{itemize}
	\item To evaluate your work, activate the evaluation:
	\begin{small}
		\begin{verbatim}
			if __name__ == '__main__':
			    test_cky()
			    #test_eval_tree()
			    #test_evaluate()
		\end{verbatim}
	\end{small}\vspace{-1cm}
\end{itemize}

\subsection{Submitting}

\begin{itemize}
	\item Submit only "\textbf{syntax.py}" file.
	Rename it "\textbf{synatx\_name1\_name2.py}".
	\item Late submission policy: You will loose 2 points after 1 day from the deadline; after that, it is a forfeit.
	\item Not attending the session will result in minus 2 points.
\end{itemize}

\subsection{Grading}

Grade = Grade\_parse + Grade\_pr\_eval + Grade\_questions + Grade\_compliance
	\begin{itemize}
		\item \textbf{Grade\_parse}: (7pts)
		\item \textbf{Grade\_pr\_eval}: (5pts)
		\item \textbf{Grade\_questions}: (2pts) + (2pts)
		\item \textbf{Grade\_compliance}: Attendance (2points) + On-time Submission (2points)
	\end{itemize}

\begin{flushright}
	Ramadhan Kareem
\end{flushright}

\end{document}
