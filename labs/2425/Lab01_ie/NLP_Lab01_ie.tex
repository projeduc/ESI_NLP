% !TEX TS-program = pdflatex
% !TeX program = pdflatex
% !TEX encoding = UTF-8
% !TeX spellcheck = en_US

\documentclass{../../../extra/aakpract/aakpract}

\settitle{Information extraction}
\setversion{Lab01}
\setyear{2024-2025}
\setauthor{Mr. Abdelkrime Aries}

\begin{document}

\maketitle

\begin{center}
	\begin{minipage}{0.8\textwidth}
		\small
		We want to extract email addresses, social media links, and phone numbers found on the "Contact Us" pages of websites in Algeria.
		The problem is that these pages present this information in various formats.
		To retrieve and unify these formats, we will use regular expressions.
	\end{minipage}
\end{center}

%We want to extract email addresses, social media links, and phone numbers found on the "Contact Us" pages of websites in Algeria.
%The problem is that these pages present this information in various formats.
%To retrieve and unify these formats, we will use regular expressions.

\section{Information to Extract}

Here, we will describe some existing variations of the information found on these pages.
This is not a complete list; therefore, you must open the pages where the system failed to identify the information to locate unrecognized formats.

\subsection{Social Media}

We are interested in the following social networks: Facebook, Twitter, YouTube, LinkedIn, and Instagram.
When an address is found, it should be returned in the following format: "\textbf{\textless network\_type\textgreater:\textless name\textgreater}".
Example: "\textbf{facebook:ESI.Page}".
We only want to retrieve specific addresses:
\begin{itemize}
	\item \textbf{LinkedIn}: Only retrieve social media links of the "company" type.
	
	For example, "https://www.linkedin.com/company/airalgerie".
	
	\item \textbf{Youtube}: Only retrieve addresses of the "channel" type.
	
	For example, "https://www.youtube.com/channel/UC4pwuVraHCCElwQ1dIl1g8Q".
\end{itemize}

\subsection{Phone Numbers}

Phone numbers come in various formats. For example: (023) 93 91 32 ; 023 93 91 32 ; 023 93.91.32 ; 023 93.91. 32 ; 023939132 ;
+213 23 93 91 32 ; + 213 23 93 91 32 ; 00 213 23 93 91 32 ; +213 (0) 23 93 91 32 ; +213 023 93 91 32 ; 023 93-91-32 ; etc.
This list is not exhaustive; failed files must be examined to detect additional formats.
The required format is: "(0XX) XX XX XX" where X is a digit.
For mobile numbers, the format will be: "(0XXX) XX XX XX".
Additional conditions:
\begin{itemize}
	\item For numbers like (023) 93 91 32/41, you must extract: (023) 93 91 32 et (023) 93 91 41. 
	There might be spaces between the digits and the "/".
	\item The same rule applies to formats like: (023) 93 91 32 à 41.
	\item Another format is: (023) 93 91 32 ou 41 ou 51. 
	In this case, extract all three numbers, replacing the last two digits accordingly.
\end{itemize}

\subsection{Email Addresses}

There are no specific conditions for email addresses.
They must be clickable; i.e. they must be shown on the page and not commented.

\section{Technical Specifications}

\begin{itemize}
	\item \textbf{Programming language}: Python.
	The program responsible for reading and evaluation is already fully implemented.
	\item \textbf{Regular Expressions \& Replacement Methods}: 
	Introduce the regular expressions and replacement patterns.
	You can use multiple patterns but should minimize the number of rules to avoid multiple passes (which would increase processing time). 
	For example, 
	\begin{small}
		\begin{verbatim}
			(re.compile(u'(\w+)@(\w+)\.(\w+)'), '\$1@\$2.\$3')
		\end{verbatim}
	\end{small}\vspace{-0.5cm}
	\item \textbf{Data}: consists of some html files and a "\textbf{ref.txt}" file containing the expected results.
	\begin{itemize}
		\item \textbf{Validation}: Shared with this assignment to test the regex.
		\item \textbf{Test}: Not shared. It is used to test that the regex are not too specific to the validation data.
	\end{itemize}
\end{itemize}

\section{Logistics}

This lab must be done in a team of no more than \textbf{2} students.
It takes \textbf{1h15} (the assignment must be submitted at the end of the session).

\subsection{Evaluation}

\begin{itemize}
	\item You have to add regular expressions to "\textbf{contacts.py}". 
	Do not modify "\textbf{ie.py}" unless you want to print only the score and not the detail. In this case, comment "\textbf{printing(comp)}" in "\textbf{main}" function.
	\item To evaluate your work:
	\begin{small}
		\begin{verbatim}
			>> python ie.py val_data/
		\end{verbatim}
	\end{small}\vspace{-1cm}
\end{itemize}

\subsection{Submitting}

\begin{itemize}
	\item Submit only "\textbf{contacts.py}" file.
	Rename it "\textbf{contacts\_name1\_name2.py}".
	\item Late submission policy: you can submit after the end of the session till midnight, but you will loose 2 points.
\end{itemize}

\subsection{Grading}

\begin{itemize}
	\item Scoring: \\Grade = (F1\_eval + F1\_test) * 10 - (0.25 * (number\_rules\_greater15 - 15)) - late\_submission\_penalty.
	\item late\_submission\_penalty = 0 if submitted at the end of the session, 2 before midnight, 2+ (0.25 * each hour).
	\item Test dataset will be shared once all participants have returned their assignments. In this case, everyone can calculate his grade.
\end{itemize}

\end{document}
